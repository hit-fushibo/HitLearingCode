\documentclass[withoutpreface,bwprint]{cumcmthesis}

\title{基于SEIR改进模型的疫情应对措施分析}

\begin{document}
\maketitle

\section{项目背景}
人类社会面临传染病的严重威胁,近几年发生的一些传染病,如非典型性肺炎(SARS)、
高致病性禽流感H5N1,甲型流感H1N1,新型冠状病毒肺炎(COVID-19)等,均对生命健康
和社会生活造成了很大的影响,如何遏制传染病爆发、缓解传染病流行,是当今社会面临的
紧迫问题,从系统科学的角度看,传染病流行时在人群中发生的一个复杂扩散过程,对这一过程进行
建模,有助于理解传染病的流行机理,认识内在规律,为干预措施的选择提供理论依据。

\section{研究内容}
\subsection{模型选择}
目前,对传染病流行过程的建模已经发展出了多种范式,概括来说,可以分为:单一群体方法、
复合群体方法和微观个体方法。
% 有一个图分别表示这几种方法
\par
考虑到目前传染病的复杂性,在这里,我们选择单一群体方法中的SEIR模型作为我们的基础模型。




\subsection{模型改进}


\subsubsection{模型拓扑结构的改进}
传统的SEIR模型将人群分为易感人群、潜伏期人群、感染人群和康复人群。人群划分较为粗糙,
与现实情况符合性较差,模型解释性欠缺。

因此,在这里我们将易感人群、潜伏期人群和感染人群进一步划分为普通易感人群、被隔离的易感人群、
普通潜伏期人群、被隔离的潜伏期人群、违背收治的感染人群和被隔离收治的感染人群。

这样一来


\subsubsection{模型参数的改进}

\subsection{模型建立}

\section{结果分析}




\end{document}